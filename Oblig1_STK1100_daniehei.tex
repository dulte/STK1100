\documentclass[a4paper,norsk, 10pt]{article}
\usepackage[utf8]{inputenc}
\usepackage{verbatim}
\usepackage{listings}
\usepackage{graphicx}
\usepackage[norsk]{babel}
\usepackage{a4wide}
\usepackage{color}
\usepackage{amsmath}
\usepackage{float}
\usepackage{amssymb}
\usepackage[dvips]{epsfig}
\usepackage[toc,page]{appendix}
\usepackage[T1]{fontenc}
\usepackage{cite} % [2,3,4] --> [2--4]
\usepackage{shadow}
\usepackage{hyperref}
\usepackage{titling}
\usepackage{marvosym }
\usepackage{subcaption}
\usepackage[noabbrev]{cleveref}
\usepackage{cite}


\setlength{\droptitle}{-10em}   % This is your set screw

\setcounter{tocdepth}{2}

\lstset{language=c++}
\lstset{alsolanguage=[90]Fortran}
\lstset{alsolanguage=Python}
\lstset{basicstyle=\small}
\lstset{backgroundcolor=\color{white}}
\lstset{frame=single}
\lstset{stringstyle=\ttfamily}
\lstset{keywordstyle=\color{red}\bfseries}
\lstset{commentstyle=\itshape\color{blue}}
\lstset{showspaces=false}
\lstset{showstringspaces=false}
\lstset{showtabs=false}
\lstset{breaklines}
\title{STK1100 Oblig 1}
\author{Daniel Heinesen, daniehei}
\begin{document}
\maketitle

\section*{Oppgave 1)}
\subsection*{a)}

$$
X = \text{Antall komponenter som testes frem til og med første defekte}
$$

Vi ser at de første $x$ vi tester, må de første $x-1$ ikke være defekte. Sjansen for at komponentene ikke er defekte er $(1-p)$. Så sannsynligheten for at dette inntreffer er $(1-p)^{x-1}$. Den siste vi tester må så være defekt, som har en sannsynelighet $p$. Dette gir oss en punktsannsynlighet

$$
p_X(x;p) = p(1-p)^{x-1}
$$

Dette er en negativ binomisk fordelig, hvor antall suksesser er $1$. Vi kan derfor også finne punktsannsynligheten gjennom denne fordelingn.

\subsection*{b)}
Den momentgenererende funksjonen for denne fordeligen er gitt ved

$$
M_X(t) = E[e^{tX}] = \sum_x e^{tx}p_X(x;p) = \sum_x e^{tx}p(1-p)^{x-1}
$$

Vi bruker så formelen for en uendelig geometrisk serie og får

$$
M_X(t) = pe^t\sum_x e^{t(x-1)}(1-p)^{x-1} = \frac{pe^t}{1-e^t(1-p)}
$$

Vi kan nå finne $E[X]$ og $E[X]$ ved hjelp av den momentgenererende funksjonen. Først finner vi:

$$
M_X'(t) = \frac{pe^t}{(1-(1-p)e^t)^2}
$$

$$
M_X''(t) = \frac{pe^t(1+(1-p)e^t)}{(1-(1-p)e^t)^3}
$$

Vi finner så at

$$
E[X] = M_X'(0) = \frac{p}{(1-(1-p))^2} = \frac{1}{p}
$$

$$
E[X^2] = M_X''(0) = \frac{p(1+(1-p))}{(1-(1-p))^3} = \frac{2-p}{p^2}
$$

Vi kan så regne ut variasjonen:

$$
V[X] = E[X^2] - E[X]^2 = \frac{2-p}{p^2} - \frac{1}{p^2} = \frac{1-p}{p^2}
$$


\subsection*{c)}

$$
Y = \text{Antall defekte komponenter i et utvalg.}
$$

Variablen over, $Y$, beskriver et eksperiment hvor vi har et antall forsøk hvor vi enten kan få suksess(defekt) eller ikke-suksess(ikke defekt).Hvorvidt komponentene er defekte eller ikke tar vi til å være uavhenging, og sannsynligheten for å finne en defekt komponent er $p$. Denne variablen forfyller alle kravene til en binomisk fordelig, og er derfor det. For denne oppgaven bruker vi at $p = 0.02$ og $n = 200$

\subsection*{d)}

Dette er programmert i python, og finnes i \textit{opp1c.py}

\begin{figure}[H]
\centering
\includegraphics[scale=0.4]{1d.png}
\caption{Den binomiske fordelingen til Y}
\end{figure}

\subsection*{e)}

Fra side $134$ i læreboken finner vi at

$$
E[Y] = np = 200\cdot 0.02 = \underline{\underline{4}}
$$
og

$$
V[Y] = np(1-p) = 4\cdot(1-0.02) = \underline{\underline{3.92}}
$$

\subsection*{f)}

En binomisk fordelig går mot en Poisson-fordeling når $n \rightarrow \infty$ og $p \rightarrow 0$. For vært eksperiment er vi lang unna, men som en tommelregel kan den binomiske fordelingen bli godt tilnærmet av Poisson-fordelingen når $n > 50$ og $np < 5$. Og siden vi har 

$$
n = 200 > 50
$$

$$
np = 4 < 5
$$

Så er en Poisson-fordeling en god tilnærming til fordelingen til $Y$.

\newpage
\subsection*{g)}

For en Poisson-fordeling 

$$
p(y;\lambda) = \frac{e^{-\lambda} \lambda^y}{y!}
$$

Hvor $\lambda$ er forventningsverdien til $Y$, som gjør at vi har

$$
\lambda = np = 4
$$

og 

$$
p(x) = \frac{e^{-4} 4^y}{y!}
$$


\subsection*{h)}

Dette er programmert i python, og finnes i \textit{opp1c.py}

\begin{figure}[H]
\centering
\includegraphics[scale=0.4]{binomialPoisson.png}
\caption{Jeg har nå valgt å plotte fordelingene som rene grafer, slik at det blir lettere å sammenlikne. Vi kan se at de ikke er helt identiske -- som vi ikke forventer --, men tilnærmingen er veldig bra.}
\end{figure}

\newpage

\section*{Oppgave 2)}

\subsection*{a)}

Vi har $X$ med sannsynlighetsfordelingen

$$
f(x) =
\begin{cases}
\theta^{-1}x^{-\frac{\theta+1}{\theta}} & x \geq 1\\
0 & x < 1
\end{cases}
$$

Den kumulative fordelingen er gitt ved

$$
F(x) = \int_0^x f(y) dy = \int_1^x \frac{y^{-\frac{\theta+1}{\theta}}}{\theta} dy 
$$

Dette reduseres til

$$
F(x) = -y^{-1/\theta}\bigg\vert_1^x = -x^{-1/\theta} - (-1) = 1 -x^{-1/\theta}
$$

\subsection*{b)}

For $\theta = 0.45$ kan vi lett finne

$$
P(2 \leq X \leq 5) = \int_2^5 f(x) dx = F(5)-F(2) = -5^{-1/0.45} + 2^{-1/0.45} = \underline{\underline{0.186}}
$$

\subsection*{c)}

Medianen til en kontinuelig fordeling er det punktet $x$ hvor arealet til fordelingen er lik på hverside. Dette kan regnes ut på to måter

$$
1) \qquad \int_{1}^x f(x) dx = \int_x^{\infty} f(x) dx
$$
$$
2) \qquad P(X \geq x) = P(X \leq x) = \frac{1}{2}
$$

Det er lett å vise at disse ble det samme. Men den enkleste er 2). Fra den får vi 

$$
P(X \geq x) = F(x) = 1-x^{-1/\theta} = \frac{1}{2}
$$

$$
\Rightarrow x = 2^{\theta}
$$

Så $x = 2^{\theta}$ er medianen til fordelingen.

\subsection*{d)}
Vi ønsker så å finne forventningsverdien til fordelingen:

$$
E[X] = \int_{-\infty}^{\infty} xf(x) dx = \int_1^{\infty} \frac{x^{-\frac{\theta +1}{\theta} + 1}}{\theta} dx
$$
$$
= \frac{1}{-\frac{1}{\theta} + 1}\frac{x^{-1/\theta + 1}}{\theta} \bigg\vert_1^{\infty} = \frac{1}{1-\theta}
$$

Det er viktig å merke seg at siden $\theta < 1/2$ så er $\lim_{x \rightarrow \infty} x^{-1/\theta + 1} = 0$, om ikke kunne integralet divergert.


\subsection*{e)}

For å finne variansen må vi først finne

$$
E[X^2] = \int_{-\infty}^{\infty} x^2f(x) dx = \int_1^{\infty} \frac{x^{-\frac{\theta +2}{\theta} + 1}}{\theta} dx
$$

$$
= \int_1^{\infty} \frac{x^{\frac{\theta -1}{\theta}}}{\theta} dx = \frac{1}{2\theta - 1}x^{\frac{2\theta - 1}{\theta}} \bigg\vert_1^{\infty} = \frac{1}{2\theta - 1}
$$

Akkurat som i deloppgaven over, vil faktumet at $\theta < 1/2$ forsikre oss om at integralet divergerer, siden $\frac{2\theta - 1 }{\theta} < 0$, som gjør at $\lim_{x \rightarrow \infty} x^{\frac{2\theta -1}{\theta}} = 0$.\\

Vi kan nå finne variansen

$$
V[X] = E[X^2] - E[X]^2 = \frac{1}{2\theta -1} - \frac{1}{(1-\theta) ^2}
$$ 


\end{document}


