\documentclass[a4paper,norsk, 10pt]{article}
\usepackage[utf8]{inputenc}
\usepackage{verbatim}
\usepackage{listings}
\usepackage{graphicx}
\usepackage[norsk]{babel}
\usepackage{a4wide}
\usepackage{color}
\usepackage{amsmath}
\usepackage{float}
\usepackage{amssymb}
\usepackage[dvips]{epsfig}
\usepackage[toc,page]{appendix}
\usepackage[T1]{fontenc}
\usepackage{cite} % [2,3,4] --> [2--4]
\usepackage{shadow}
\usepackage{hyperref}
\usepackage{titling}
\usepackage{marvosym }
\usepackage{subcaption}
\usepackage[noabbrev]{cleveref}
\usepackage{cite}


\setlength{\droptitle}{-10em}   % This is your set screw

\setcounter{tocdepth}{2}

\lstset{language=c++}
\lstset{alsolanguage=[90]Fortran}
\lstset{alsolanguage=Python}
\lstset{basicstyle=\small}
\lstset{backgroundcolor=\color{white}}
\lstset{frame=single}
\lstset{stringstyle=\ttfamily}
\lstset{keywordstyle=\color{red}\bfseries}
\lstset{commentstyle=\itshape\color{blue}}
\lstset{showspaces=false}
\lstset{showstringspaces=false}
\lstset{showtabs=false}
\lstset{breaklines}
\title{STK1100 Oblig 2}
\author{Daniel Heinesen, daniehei}
\begin{document}
\maketitle

\section*{Oppgave 1)}

\subsection*{a)}
Vi har en simultan sannsynlighetsfordelig

\begin{equation}
f(x,y) = 
\begin{cases}
k(x-y) & 0 \leq y \leq x \leq 1 \\
0 & \text{ellers}
\end{cases}
\label{eq:f(x,y)}
\end{equation}

for å finne $k$ kan vi bruke at fordeligen må være normalisert. Dette betyr at 

\begin{equation}
I = \int_0^1 \int_0^x k(x-y) dy dx = 1
\end{equation}

Vi kan regne dette ut:

$$
I = \int_0^1 \int_0^x k(x-y) dy dx = k\int_0^1 \left[xy-\frac{1}{2}y^2\right]_0^x dx
$$
$$
= k\int_0^1 \frac{1}{2}x^2 dx = k\frac{1}{6}x^3\bigg|_0^1 = \frac{k}{6} = 1
$$

Dette gir oss at

\begin{equation}
\underline{\underline{k = 6}}
\end{equation}


\subsection*{b)}
Vi ønsker å finne $P(2Y \leq X)$. Dette er sannsynligheten at om vi velger en tilfeldig $x$ og en tilfeldig $y$, så er $2Y \leq X \Leftrightarrow Y \leq X/2$. Vi finner sannsynligheten ved

$$
P(2Y \leq X) = \int_{0}^1\int_0^{x/2}  k(x-y) dydx =k \int_0^{1}\left(xy - \frac{1}{2}y^2\right)\bigg |_0^{x/2}dx = k\int_0^{x/2} \frac{1}{2}x^2 - \frac{1}{8}x^2 dx
$$
$$
= k\frac{1}{8}x^3\bigg|_0^1 = \frac{k}{8} 
$$
Setter vi inn verdien for $k$ det endelig svaret:

\begin{equation}
\underline{\underline{P(2Y \leq X) = \frac{3}{4}}}
\end{equation}

\subsection*{c)}
Vi ønsker så å finne den marginale sannsynlighetsfordeligen for $X$. Vi ser først på hvor $0 \leq y \leq x \leq 1$. For å få den marginale sannsynlighetsfordeligen må vi integrere sannsynlighetsfordeligen over alle mulige verdier av $y$, som er $0 \leq y \leq x$, vi får da

\begin{equation}
f_X(x) = \int_0^x k(x-y)dy = k\left(xy-\frac{1}{2}y^2\right)\bigg|_0^x = \frac{k}{2}x^2 = \underline{{3x^2}}
\end{equation}

For vi ser nå på de øvrige funksjonene. Her vil $f(x,y) = 0$, som fører til at også $f_X(x) = 0$. Så vi ender opp med at

\begin{equation}
f_X(x) = 
\begin{cases}
3x^2 & 0 \leq x \leq 1 \\
0 & \text{ellers}
\end{cases}
\end{equation}


\subsection*{d)}

Vi ønsker så å finne den marginale sannsynlighetsfordeligen for $Y$. Vi starter også her med å se på tilfellene hvor $0 \leq y \leq x \leq 1$. Vi ønsker å integrere $f(x,y)$ over alle de mulige verdiene av $x$, hvilket er $y\leq x \leq 1$:

\begin{equation}
f_Y(x) = \int_y^1 k(x-y)dy = k\left(\frac{1}{2}x^2 - yx\right)\bigg|_y^1 = k\left(\frac{1}{2}y^2 - y + \frac{1}{2}\right) = \underline{3y^2 - 6y + 3}
\end{equation}

Akkurat som i deloppgaven over vil $f(x,y) = 0$ for de øvrige tilfellen, noe som gjør at $f_Y(x) = 0$ her. Så vi ender opp med:

\begin{equation}
f_X(x) = 
\begin{cases}
3y^2 - 6y + 3 & 0 \leq y \leq 1 \\
0 & \text{ellers}
\end{cases}
\end{equation}

\subsection*{e)}

For å sjekke om $X$ og $Y$ er uavhenginge kan vi gange samme de marginale sannsynlighetsfordelingene og se om vi før den orginale sannsynlighetsfordelingen \eqref{eq:f(x,y)}:

\begin{equation}
f_X(x)\cdot f_Y(y) = 3x^2\cdot(3y^2 -6y +3) = 9x^2y^2 -6x^2y + 9x^2 \neq f(x,y)
\end{equation} 

Siden $f_X(x)\cdot f_Y(y) \neq f(x,y) $ er $X$ og $Y$  \textbf{ikke} uavhenginge.


\section*{Oppgave 2)}

\textit{Koden finnes som opp2b.py. Denne koden vil lage alle plottene som er brukt i denne oppgaven.(Alle plotten vil lages og vises når man kjører programmet, så vær forberedt på den når den som retter dette kjører programmet)}


\subsection*{a)}

Vi har flere identiske stokatiske variabler $X_i$ som følger: 

\begin{equation}
E(X_i) = \mu, \qquad V(X_i) = \sigma^2
\end{equation}

Vi innfører så at 

\begin{equation}
\overline{X}_i = \frac{1}{n}\sum_{i = 1}^n X_i
\label{eq:manyMean}
\end{equation}

Vi har også det standariserte gjennomsnittet

\begin{equation}
Z_n = \frac{\overline{X}_i - \mu}{\sigma/\sqrt{n}} = \sqrt{n}\frac{\overline{X}_i - \mu}{\sigma}
\label{eq:stanardMean}
\end{equation}

Vi kommer til å bruke at for forventingsverdien til flere uavhengige stokastiske variabler er

\begin{equation}
E\left(a + \sum_{i = 1}^n b_i X_i\right) = a + \sum_{i = 1}^n b_i E(X_i)
\label{eq:Elinear}
\end{equation}

Og for variansen er gjelder

\begin{equation}
V\left(a + \sum_{i = 1}^n b_i X_i\right) = \sum_{i = 1}^n b_i^2 V(X_i)
\label{eq:Vlinear}
\end{equation}

Vi kan fra dette se at for de uavhengige stokastiske variablene $X_i$ gjelder:

\begin{equation}
E(\overline{X}_i) = E\left(\frac{1}{n}\sum_{i = 1}^n X_i\right) = \frac{1}{n}\sum_{i = 1}^n E(X_i) = \frac{1}{n}n\mu = \underline{\underline{\mu}}
\end{equation}

og

\begin{equation}
V(\overline{X}_i) =  V\left(\frac{1}{n}\sum_{i = 1}^n X_i\right) = \frac{1}{n^2}\sum_{i = 1}^n  V(X_i) = \frac{1}{n^2}n\sigma^2 = \underline{\underline{\frac{\sigma^2}{n}}}
\end{equation}

Og for det standariserte gjennomsnittet gjelder

\begin{equation}
E(Z_n) = E\left(\sqrt{n}\frac{\overline{X}_i - \mu}{\sigma}\right) = \sqrt{n}\left(\frac{E(\overline{X}_i)}{\sigma} - \frac{\mu}{\sigma}\right) = 
\sqrt{n}\left(\frac{\mu}{\sigma} - \frac{\mu}{\sigma}\right) = \underline{\underline{0}}
\end{equation}

og

\begin{equation}
V(Z_n) = V\left(\sqrt{n}\frac{\overline{X}_i - \mu}{\sigma}\right) = n\frac{V(\overline{X}_i)}{\sigma^2} = n\frac{\sigma^2/n}{\sigma^2} = \underline{\underline{1}}
\end{equation}

\subsection*{b)}

Vi skal her bestemme forventingsverdien og variansen til de 3 følgende fordelingene:

\subsubsection*{Uniform Fordeling:}

Vi har en uniform fordeling:

\begin{equation}
f(x) =
\begin{cases}
\frac{1}{2} & -1\leq x  \leq 1\\
0 & ellers
\end{cases}
\end{equation}

Vi kan regne ut fordelingen:

\begin{equation}
\mu = E(X_i) = \int_{-1}^1 xf(x) dx =\int_{-1}^1 x\frac{1}{2} dx = \underline{0} 
\end{equation}

Siden integranden er en 'odd' funksjon. For å regne ut variansen regner vi først ut:

\begin{equation}
E(X_i^2) = \int_{-1}^1 x^2f(x) dx = \int_{-1}^1 \frac{1}{2}x^2 dx =\frac{1}{6}x^3\bigg|_{-1}^1 = \frac{1}{3} 
\end{equation}

Vi har da at variansen er

\begin{equation}
\sigma^2 = V(X_i) = E(X_i^2) - E(X_i)^2 = \underline{\underline{\frac{1}{3}}}
\end{equation}

\subsubsection*{Gammafordeling:}

Vi har gammafordelingen:

\begin{equation}
f(x) =
\begin{cases}
\frac{1}{\sqrt{\pi x}}e^{-x} & x > 0\\
0 & ellers
\end{cases}
\end{equation}

Dette er en gammafordeling med $\alpha = 1/2$ og $\beta = 1$. Vi vet hva definisjonene på forventingsverdien og variansen er for gammafordelinger, så vi kan fort finne at for denne fordelingen er:

\begin{equation}
\mu = \alpha\beta = \underline{\underline{\frac{1}{2}}}
\end{equation}

og 

\begin{equation}
\sigma^2 = \alpha\beta^2 = \underline{\underline{\frac{1}{2}}}
\end{equation}


\subsubsection*{Bernoullifordeling:}

Vi har en Bernoullifordeling med punktfordelingen $p(x) = P(X = x)$, med en $p = 0.75$, som gir at $p(1) = 0.75$ og $p(0) = 0.25$. Vi vet også de generelle formelene for forventingsverdien og variansen til Bernoullifordelinger. Vi finner så at

\begin{equation}
\mu = p = \underline{\underline{0.75}}
\end{equation}

og

\begin{equation}
\sigma^2 = p(1-p) = \frac{1}{4}\cdot \frac{3}{4} = \underline{\underline{\frac{3}{16}}}
\end{equation}

\subsection*{c)}

Vi forventer å se at det normerte histogrammet av observasjonene nærmer seg sannsynlighetstettheten til $Z_n$ blir større og større. Teller vi opp antall observasjoner med en viss verdi/innen for et vist intervall vil forvente at jo høyere sannsynlighet for å finne denne observasjonen, jo flere tilfeller vil vi observere, og vis versa. Med nok antall observasjoner vil observasjonen alltid legge seg etter fordlingen til $Z_n$. Siden fordelingen til $Z_n$ er normalisert, så må vi også normere histogrammet for at det skal likne på fordelingen til $Z_n$.

\subsection*{d)}


\begin{figure}[H]
\centering
\includegraphics[scale=0.4]{histoUniform5.png}
\caption{Histogram for en uniform fordeling med $n = 5$}
\label{fig:histoUniform5}
\end{figure}


Vi ser over et histogram med de standariserte gjennomsnittene for en uniform fordeling. Siden vi regnet ut standardiserte gjennomsnitt av observasjoner av fordelingene, så vil vi etter sentralgrenseteoremet forvente at $Z_n$ og dermed histogrammet skal gå mot en normalfordeling. Vi kan se at histogrammet over er veldig nært en normalfordeling. Rett rundt $\mu = 0$ er det noen ekstra topper, men ellers er fordelingen ganske normalfordelt. 

\subsection*{e)}

\textit{Verdiene under finnes i utskrifen til python-programmet.}


Tabell for sannsynlighetene til en standardnomalfordelt variable:\\

\begin{table}[H]
\centering
\begin{tabular}{|l|c|}
\hline
Interval & Sannsynlighet\\
\hline
$- \infty$ til $-2.5$ & 0.0062\\
$- 2.5$ til $-2.0$ & 0.0165\\
$- 2.0$ til $-1.5$ & 0.0441\\
$- 1.5$ til $-1.0$ & 0.0919\\
$- 1.0$ til $-0.5$ & 0.1499\\
$- 0.5$ til $0.0$ & 0.1915\\

$ 0.0$ til $0.5$ & 0.1915\\
$ 0.5$ til $1.0$ & 0.1499\\
$ 1.0$ til $1.5$ & 0.0919\\
$ 1.5$ til $2.0$ & 0.0441\\
$ 2.0$ til $2.5$ & 0.0165\\
$ 2.5$ til $\infty$ & 0.0062\\ \hline
\end{tabular}
\caption{Sannsynlighetene til en standardnomalfordelt variable}
\label{tab:stand}
\end{table}


\subsection*{f)}

På samme måte som vi fant verdiene for en standardnomalfordelt variable, kan vi finne verdiene for vår fordelinge av det standardiserte gjennomsnittet av den uniforme fordelingen:

\begin{table}[H]
\centering
\begin{tabular}{|l|c|}
\hline
Interval & Sannsynlighet\\
\hline
$- \infty$ til $-2.5$ & 0.0044\\
$- 2.5$ til $-2.0$ & 0.0171\\
$- 2.0$ til $-1.5$ & 0.0457\\
$- 1.5$ til $-1.0$ & 0.0926\\
$- 1.0$ til $-0.5$ & 0.1486\\
$- 0.5$ til $0.0$ & 0.1905\\

$ 0.0$ til $0.5$ & 0.1931\\
$ 0.5$ til $1.0$ & 0.1494\\
$ 1.0$ til $1.5$ & 0.0901\\
$ 1.5$ til $2.0$ & 0.0469\\
$ 2.0$ til $2.5$ & 0.0175\\
$ 2.5$ til $\infty$ & 0.0041\\ \hline
\end{tabular}
\caption{Sannsynlighetene til en unifrom fordeling med $n = 5$}
\label{tab:uniform5}
\end{table}

Vi kan se at disse verdiene er veldig nære de vi fant i deloppgaven over \ref{tab:stand}. Vi kan finne absoluttveriden til forskjellen mellom fordelingen:

\begin{table}[H]
\centering
\begin{tabular}{|l|c|}
\hline
Interval & Forskjell\\
\hline
$- \infty$ til $-2.5$ & 0.0018\\
$- 2.5$ til $-2.0$ & 0.0056\\
$- 2.0$ til $-1.5$ & 0.00164\\
$- 1.5$ til $-1.0$ & 0.0008\\
$- 1.0$ til $-0.5$ & 0.0012\\
$- 0.5$ til $0.0$ & 0.0010\\

$ 0.0$ til $0.5$ & 0.0016\\
$ 0.5$ til $1.0$ & 0.0005\\
$ 1.0$ til $1.5$ & 0.0015\\
$ 1.5$ til $2.0$ & 0.0028\\
$ 2.0$ til $2.5$ & 0.0010\\
$ 2.5$ til $\infty$ & 0.0021\\ \hline
\end{tabular}
\caption{Forskjellen til en unifrom fordeling med $n = 5$}
\label{tab:forskjellUniform5}
\end{table}

Vi ser her at forskjellen mellom standardnormalfordelingen og vår fordeling er svært liten, som vi forventet ut i fra histogrammet i \ref{fig:histoUniform5}.\\

\textit{Jeg kommer til å ha sannsynligheten og forskjellen fra normalfordelingen i de samme tabellene i resten av oppgaven.}

\subsection*{g)}



\end{document}


